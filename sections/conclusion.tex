\documentclass[../main.tex]{subfiles}

\begin{document}

\chapter{Заключение}

\section{Результаты}

    Изучены теоретические основы расчётов дисперсионных соотношений носителей заряда в
    гетероструктурах с квантовыми ямами на основе $HgCdTe$.

    Разработана программная реализация этих численных моделей.
    Это программное обеспечение позволяет расчитывать зонные спектры таких структур при 
    всех возможных параметрах и произвольном профиле распределения состава по структуре.

    Произведённые с его помощью расчёты совпадают с ранее полученными результатами, а 
    в некоторых случаях позволяют предсказывать и новые эффекты.
    
\section{Планы}

    Особый интерес представляют структуры содержащие туннельно-связанные квантовые ямы.
    Рост таких структур позволяет в существенной мере варьировать зонный спектры
    тем самым варьируя и величину порога оже-рекомбинации.

    Другим интересным эффектом, требующим детального рассмотрения является воздействие 
    магнитного поля на носителей заряда в полупроводниках. В данном случае это представляет
    особый интерес, поскольку характерный масштаб энергии уровней Ландау может многократно 
    превышать ширину запрещённой зоны и тем более расстояние между подзонами.
\end{document}