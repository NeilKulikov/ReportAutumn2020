\documentclass[../main.tex]{subfiles}

\begin{document}

\chapter{Заключение}

\section{Результаты}

    Изучены теоретические основы расчётов дисперсионных соотношений носителей заряда в
    гетероструктурах с квантовыми ямами на основе $HgCdTe$.

    Разработана программная реализация этих численных моделей.
    Это программное обеспечение позволяет расчитывать зонные спектры таких структур при 
    всех возможных параметрах и произвольном профиле распределения состава по структуре.

    Произведённые с его помощью расчёты совпадают с ранее полученными результатами, а 
    в некоторых случаях позволяют предсказывать и новые эффекты.
    
    При помощи сторонней программы были произведены расчёты дисперсилнных соотношений 
    плазмонов в квантовых ямах. Продемонстрирована возможность управления процессами
    поглощения/излучения плазмонами при помощи изменения концентрации носителей 
    заряда.
\end{document}