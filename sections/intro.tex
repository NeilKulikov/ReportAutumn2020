\documentclass[../main.tex]{subfiles}

\begin{document}

\chapter{Введение}

Одной из наиболее актуальных проблем современной прикладной физики является
создание источников когерентного излучения терагерцового (ТГц) диапазона. Такие
источники могли бы использоваться во множестве медицинских приложений, ввиду достаточно 
малого поглощения этого излучения тканями человека, что позволило бы разработать 
новые неинвазивные методы диагностики онкологии и иных заболеваний; другим возможным 
применением может стать спектроскопия сложных органических соединений, поскольку 
они имеют вращательные и колебательные степени свободы, имеющие соответствующие частоты 
лежащие именно в терагерцовом диапазоне.

В настоящее время имеется несколько способов генерации подобного излучения, которые, однако, 
имеют много недостатков. Одним из классов таких приборов являются квантово-каскадные лазеры (ККЛ).
Они демонстрируют превосходные характеристики (высокий КПД, высокий уровень когерентности)
в диапазоне 1-5 THz и выше 15 THz \cite{Intro1}. Однако большая часть таких лазеров создается на основе 
полупроводников типа A3B5 (GaAs, PbSb или InP), которые имеют высокое поглощение на оптических 
фононах в диапазоне 5-15 THz. QCL на основе GaN подступают к спектральному диапазону 5 - 15 THz со
стороны низких частот (относительно частот оптических фононов), но их рабочие характеристики 
требуют значительного улучшения \cite{Intro7}.

Альтернативой ККЛ являются лазеры на основе межзонных переходов в узкозонных полупроводниковых структурах.
Такие лазеры намного проще в изготовлении, а также позволяют наблюдать стимулированное излучение и генерацию
в диапазоне 5-15 THz. Кроме того, их 
отличительной чертой является возможность перестройки частоты в достаточно широком диапазоне, за счет 
изменения температуры. Однако ограничивающим
фактором является процесс безызлучательной оже-рекомбинации. При этом традиционно ожидается, что вероятность 
межзонной оже-рекомбинации растёт с уменьшением ширины запрещённой зоны, что может затруднить генерацию
стимулированного излучения.

Оже-рекомбинация представляет собой безызлучательный трёхчастичный процесс. По типу носителей заряда, участвующих
в процессе он делится на CCHC и HHCH процессы (процесс с участием двух электронов и дырки и процесс с двумя дырками
и электроном соответственно). В ходе этого процесса пара носителей с противоположным зарядом рекомбинирует и
передаёт энергию и импульс третьему. В силу ограничений, накладываемых законами сохранения энергии и импульса,
 этот процесс является пороговым.
Поэтому температура сильно влияет на темп таких процессов, а значит и на эффективность лазеров.

Повлиять на это можно, варьируя материалы/структуры и изменяя тем самым дисперсионные соотношения в них. В частности,
существуют законы дисперсии, в которых такие процессы принципиально запрещены законами сохраниения (к примеру Дираковский 
закон дисперсии для массовых частиц). С другой стороны возможны структуры, в которых энергетический порог таких оже-процессов 
стремитсяк нулю (к примеру Дираковский закон дисперсии безмассовых частиц).

Спектральный диапазон 5 - 15 THz к настоящему моменту частично 
перекрыт лишь диодными лазерами на основе халькогенидов свинца-олова, которые обеспечивают длины 
волн излучения вплоть до 46.5 $\mu m$ \cite{Intro8}. Фактор, который снижает эффективность оже-рекомбинации в PbSnSe(Te) \cite{Intro1}, \cite{Intro9}
- симметрия между законами дисперсии носителей в зоне проводимости и в валентной зоне. Однако их рабочие характеристики 
ограничены технологией роста: существуют труднопреодолимые проблемы в реализации квантовых ям (КЯ) для твердых растворов 
PbSnSe(Te) и остаточная концентрация носителей остается на высоком уровне $10^{17} \text{cm}^{-3}$. К тому же технология производства подобных
полупроводников пока не отработана в достаточной мере, что не позволяет выращивать структуры с квантовыми ямами и единственной возможностью
влиять на дисперсионное соотношение и ширину запрещенной зоны является изменение состава.

Существуют альтернативные полупроводниковые системы, которые позволяют
приблизиться к дираковскому закону дисперсии, но с конечной шириной запрещенной зоны. 
Как было показано в многочисленных работах \cite{Rumyantsev:IOP:2018},\cite{Rumyantsev:2019}, одна из таких систем - гетероструктуры с КЯ на основе Hg(Cd)Te/CdHgTe. 
В отличие от графена, в структурах на основе HgCdTe (КРТ) с КЯ можно перестраивать ширину запрещенной зоны путем изменения ширины КЯ и 
содержания Cd в ней. Современная молекулярно-лучевая эпитаксия (МЛЭ) обеспечивает высокое качество эпитаксиальных пленок КРТ не только на подложках CdZnTe, 
но и на «альтернативных» подложках GaAs \cite{Varavin:2003}. Высокое качество эпитаксиальных структур HgCdTe, выращенных на GaAs подложках, было подтверждено в ходе исследований 
фотопроводимости (ФП) и фотолюминесценции (ФЛ) в среднем и дальнем инфракрасном диапазонах ($λ$ = 15-30 $\mu m$). Было получено 
стимулированное излучение (СИ) в КРТ структурах с КЯ на длине волны до 20.3 $\mu m$ \cite{Rumyantsev:IOP:2018}, в то время как ранее лазерная генерация в HgCdTe была 
продемонстрирована лишь в коротковолновой области среднего инфракрасного диапазона спектра (на длинах волн 2 - 5 $\mu m$). 

Особый интерес представляют процессы с участием плазмонов в квантовых ямах HgCdTe. Двумерные плазмоны терагерцового диапазона сами по себе демонстрируют 
некоторые удивительные свойства: высокий коэффициент усиления за счет низкой скорости плазмона, отсутствие необходимости создания внешней волноводной 
структуры \cite{bergman2003surface},\cite{berini2012surface},\cite{oulton2009plasmon}, \cite{khurgin2012practicality}. HgCdTe квантовые ямы являются 
отличным объектом для демонстрации таких свойств. Кроме того их производства является сравнительно отработанной технологией в сравнении с 
технологией производства графена - другого перспективного материала.

В таком случае идеи лазеров на основе квантовых ям и плазмонов могут быть объединены \cite{kapralov2019feasibility} т.к. квантовые ямы поддерживают собственные 
плазмонные моды, что обеспечивает лучшее удержание поля в сравнении с поверхностными плазмонами на границе металл-изолятор \cite{stern1967polarizability}.
В этом подходе межзонные переходы будут обеспечивать усиление плазмонов. Однако это возможно лишь в узкозонных или бесщелевых полупроводниках
из-за сравнительно низкой концентрации носителей заряда и низких частот плазмонов. HgCdTe как полупроводник считается наилучшим кандидатом 
для реализации данной идеи.

% Без учета пространственной дисперсии
% КОММЕНТАРИЙ НЕВЕРЕН - "пренебрегая пространственной дисперсией"

% Нет учета взаимодействия с плазмонами
% Нет вклада перехода между валентными зонами
% Нет легирования

Проблема усиления плазмонов таких структурах уже была рассмотрена в статье \cite{kapralov2019feasibility}. Авторы рассматривали упрощенную зонную модель
BHZ (Bernevig-Hughes-Zhang) \cite{bernevig2006quantum} и зависимость коэффициента усиления от концентрации носителей в квантовой яме, пренебрегая 
легированием, переходами между валентными зонами и взаимодействием с плазмонами. В качестве одного из основных результатов был рассмотрен 
вопрос пороговой концентрации носителей заряда.

В данной работе проводится расчёт энергетических зонных спектров и соответствующих волновых функций узкозонных гетероструктур с квантовыми
ямами на основе твёрдых растворов $Hg_{1-x}Cd_{x}Te$. Вычисляется спектр двумерных плазмонов $\omega(\vec q)$ и зависимость усиления
(которые и определяются мнимой части частоты $\Im \omega$), производится анализ для различных концентраций носителей заряда без приближений, 
использованных в \cite{kapralov2019feasibility}. В частности рассматривается вопрос влияния легирования.

\end{document}