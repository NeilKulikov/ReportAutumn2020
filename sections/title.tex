\documentclass[../main.tex]{subfiles}

\begin{document}
\begin{titlepage}
    \begin{center} 
      \small{\textbf{МИНИСТЕРСТВО НАУКИ И ВЫСШЕГО ОБРАЗОВАНИЯ РОССИЙСКОЙ ФЕДЕРАЦИИ}}
    \end{center}

    \begin{center}
    Федеральное государственное автономное образовательное учреждение высшего образования «Национальный исследовательский Нижегородский государственный университет им. Н.И. Лобачевского» (ННГУ)
    
    \vspace{0.0cm}
    \textbf{Высшая школа общей и прикладной физики}
    
    \end{center}
    
    \begin{center}
    {\bf{Отчёт о прохождении производственной практики}}

    {
      \small{(научно-исследовательская работа)}
    }

    \vspace{0.1cm}

    \bf{Расчёт дисперсионных соотношений в гетероструктурах с квантовыми ямами на основе
    твёрдых растворов HgCdTe}
    \end{center}
    
    \begin{flushright}
    студента 1 курса магистратуры по
    
    направлению подготовки 03.04.02 «Физика»,
    
    профиль – физика конденсированного состояния,
    
    Куликова Никиты Сергеевича

    \vspace{0.3 cm}
    \underline{Руководитель практики от ННГУ:}
    
    Профессор, доктор физико-математических наук:
    
    \begin{minipage}[t][0.1 cm][t]{2in}
    %  \begin{minipage}[t][2\mytextsize][t]{2in} % размер minipage равен удвоенному размеру основного шрифта
      \underline{\hspace{2in}}\\ % линия подчёркивания на два дюйма
      \centering
    %   \vspace{\mytextsize} % отступ minipage для выравнивания линии подчёркивания с базовой линией остального текста
    \end{minipage}~В.Я. Алёшкин
    
    % \vspace{6\mytextsize}
    \vspace{0.3 cm}
     \underline{Руководитель практики от ИФМ РАН:}
    
    Научный сотрудник ИФМ РАН,

    Кандидат физико-математических наук
    
    \begin{minipage}[t][0.1cm][t]{2in} % размер minipage равен удвоенному размеру основного шрифта
    
    % \begin{minipage}[t][2\mytextsize][t]{2in} % размер minipage равен удвоенному размеру основного шрифта
      \underline{\hspace{2in}}\\ % линия подчёркивания на два дюйма
      \centering
      \vspace{0.1 cm} % отступ minipage для выравнивания линии подчёркивания с базовой линией остального текста
    \end{minipage}~М.С. Жолудев
    
    \vspace{0.3 cm}
    
    \underline{Декан ВШОПФ:}
    
    Кандидат физико-математических наук
    
    \begin{minipage}[t][0.1 cm][t]{2in} % размер minipage равен удвоенному размеру основного шрифта
      \underline{\hspace{2in}}\\ % линия подчёркивания на два дюйма
      \centering
      \vspace{0.1 cm} % отступ minipage для выравнивания линии подчёркивания с базовой линией остального текста
    \end{minipage}~Е. Д. Господчиков
    
    \end{flushright}
    
    \vspace{0.1cm}
    
    \begin{center}
        Нижний Новгород
    
        2020 г.
    \end{center}
    
    \end{titlepage}
\end{document}